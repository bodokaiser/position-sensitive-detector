\section{Analogue signal processing}

In the former section we explained how the photo currents $I_{X1},I_{X2},I_{Y1},I_{Y2}$ from the anode output terminals of the \gls{psd} relate to the center-of-mass position of the incident optical signal.
We closed the section with an equivalent circuit for the output terminals.
In the present section we design the analogue signal processing required to convert the photo currents to a voltage signal that relates to the center-of-mass position of the optical signal.

\subsection{Preamplifier}

The first stage of our analogue signal processing converts the current signal of each of the output terminals to a voltage signal.
Current to voltage conversion is typically implemented using a transimpedance amplifier circuit.
\begin{figure}[H]
	\centering
	\begin{circuitikz}
		\draw
			(0, 0)
				node[op amp] (opamp) {}
			(opamp.-)
				to[short] (-1, 0.5)
				to[american current source, invert, label=$I$] (-4, 0.5)
				to[short] (-4, -1.5)
				node[ground] {}
			(opamp.-)
				to[short, *-] ++(0, 1.5)
				coordinate (left)
				to[resistor, label=$R$] (left -| opamp.out)
				to[short, -*] (opamp.out)
			(opamp.+)
				to[short] +(0, -1)
				node[ground] {}
			(opamp.out)
				to[short] (3, 0)
				node[ocirc, label=$V$] {}
	;
	\end{circuitikz}
	\caption{Transimpedance amplifier circuit.}\label{fig:circuit_transimpedance_amp}
\end{figure}
\Cref{fig:circuit_transimpedance_amp} illustrates a transimpedance amplifier circuit.
An ideal operational amplifier outputs a voltage proportional to the voltage difference between the inverting $-$ and non-inverting input $+$.
The output of the operational amplifier is coupled by a feedback resistor $R$ with the non-inverting input.