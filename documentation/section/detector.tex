\section{Detector}

Motivate the choice of the \gls{psd} to be an improved tetra-lateral PSD photodiode compared to i.e.\ quadrant diodes and explain the working principle~\cite{Noorlag74}.
We have chosen the \gls{s5990} as the position-sensitive detector.
The parameters of the \gls{s5990} are reported in \Cref{tab:s5990}.
\begin{table}
	\label{tab:s5990}
	\centering
	\begin{tabular}{lllll}
		\toprule
		\multirow{2}[3]{*}{Parameter} &
		\multirow{2}[3]{*}{Symbol} &
		\multicolumn{2}{c}{Values} &
		\multirow{2}[3]{*}{Unit} \\
		\cmidrule(lr){3-4}
		& & Typical & Maximum & \\
		\midrule
		Dark current & $I_\text{dark}$ & \num{0.5} & \num{10} & \si{\nano\ampere}\\
		Interelectrode resistance & $R_\text{ie}$ & \num{7} & \num{15} & \si{\kilo\ohm}\\
		Terminal capacitance & $C_t$ & \num{150} & \num{300} & \si{\pico\farad}\\
		Position resolution & $\Delta x$ & \num{0.7} & & \si{\micro\meter}\\
		\bottomrule	
	\end{tabular}
	\caption{Important parameters of the \gls{s5990} extracted from the datasheet~\cite{HamamatsuPSD}.}
\end{table}

For high speed applications one usually connects the cathode to a reverse bias voltage.
However, in our case we are interested in a high spatial resolution. In this case ...ref... the \gls{psd} cathode should be connected to ground.

The output terminals of the \gls{psd} can be modelled as a current source with internal resistance and capacitance~\cite{HamamatsuPSD}.
\begin{circuitikz}
	\draw (0, 0)
		to[american current source, l=$I_\text{photo}$] (0, 4)
		to[ammeter] (4, 4) -- (4, 0)
		to[lamp] (0, 0)
		;
\end{circuitikz}
