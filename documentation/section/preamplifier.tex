\section{Preamplifier}

The photocurrents created by our detector are in the range of microampere where they are vulnerable to noise.
Using a preamplifier, we can increase the amplitude of the signal for an improved signal-to-noise ratio.
The typical photocurrent preamplifier is based on the transimpedance (current-to-voltage) amplifier design using a voltage-feedback operational amplifier.
Converting the current to a voltage signal has the benefit that the voltage signal can be easily visualized with an oscilloscope.
Furthermore, the voltage-feedback operational amplifier design appears to be more common than the current-feedback operational amplifier, as manufacturers offer much more choice and they are more prominent in the literature.
That said, current-feedback operational amplifiers are reported to be a viable solution for high-speed and high-bandwidth applications, see Ref.~\cite[p.~110]{Jung05} for an overview of the benefits of current-feedback amplifiers and Ref.~\cite[Ch.~9]{Carter17} for a comparison with voltage-feedback amplifiers.

In the following, we will always refer to the voltage-feedback operational amplifier if not noted otherwise.
Moreover, we expect the reader to be familiar with the foundations of the operational amplifier.
Well-written introductions to this topic can be found in \cite[Ch.~1]{Jung05} and \cite[Ch.~3]{Carter17}.

% TODO: mention finite loop gain error (Jung, p. 13)
% TODO: table with op amp parameters specified in the datasheet and their relevance for our application
% TODO: noise model (Jung, p. 80)

\subsection{Basic design}

The basic design of a transimpedance amplifier is illustrated in~\Cref{fig:basic_transimpedance}.
Ignoring imperfections of the photodiode, we can represent the photodiode as a current source.
The non-inverting input of the operational amplifier is connected to ground.
The inverting input is coupled with the output of the operational amplifier via a feedback impedance $Z_f$.
\begin{figure}[H]
	\centering
	\begin{circuitikz}
		\draw (0, 0) node[op amp](opamp){};
		\draw (opamp.+) -- +(0, -1) node[ground](gnd1){};
		\draw (opamp.out) -- +(1, 0) node[ocirc, label=$V_\text{out}$]{};
		\draw (opamp.-) node[circ]{} -- +(-2, 0) node(node1){} to[current source, invert, l=$I_\text{in}$] (node1 |- gnd1) node[ground]{};
		\draw (opamp.-) |- (-1, 2) to[generic, l=$Z_f$] (1,2) -| (opamp.out);
	\end{circuitikz}
	\caption{Basic transimpedance amplifier using voltage feedback operational amplifier.}\label{fig:basic_transimpedance}
\end{figure}
The ideal operation amplifier has zero input current, thus, in the node of the inverting input of the operational amplifier Kirchhoff's law states that the current going through the feedback impedance has to cancel the current of the current source $I_\text{in}$.
The current through the feedback impedance $Z_f$ can be expressed in terms of the feedback impedance $Z_f$ and the output voltage $V_\text{out}$ of the operational amplifier through the use of Ohm's law, yielding,
\begin{equation}
	V_\text{out}=-Z_fI_\text{in}
	\label{eq:transimpedance}.
\end{equation}
Given a maximum current $I_\text{in}$ and a desired maximum output voltage $V_\text{out}$, \Cref{eq:transimpedance} determines the feedback impedance.
Limitations arise for real operational amplifiers where the output voltage is limited to be below the supply voltage of the operational amplifier.


Aside from photodiode applications, it is more common to find the inverting (voltage-to-voltage) operational amplifier in the literature.
By using the source transformation on the current source with a parallel impedance in the transimpedance amplifier circuit, we can recover the inverting operational amplifier circuit, and thereby easily use results obtained for the inverting amplifier design.
\begin{figure}[H]
	\begin{subfigure}[t]{.5\textwidth}
		\centering
		\begin{circuitikz}
			\draw (0, 0) node[op amp](opamp){};
			\draw (opamp.+) -- +(0, -1) node[ground](gnd1){};
			\draw (opamp.out) -- +(1, 0) node[ocirc, label=$V_\text{out}$]{};
			\draw (opamp.-) node[circ]{} -- +(-3, 0) node(node1){} to[current source, invert, l=$I_\text{in}$] (node1 |- gnd1) node[ground]{};
			\draw (opamp.-) node[circ]{} -- +(-1.5, 0) node(node1){} to[generic, invert, l=$Z_\text{in}$] (node1 |- gnd1) node[ground]{};
			\draw (opamp.-) |- (-1, 2) to[generic, l=$Z_f$] (1,2) -| (opamp.out);
		\end{circuitikz}
		\caption{Transimpedance amplifier.}
	\end{subfigure}
	\begin{subfigure}[t]{.5\textwidth}
		\centering
		\begin{circuitikz}
			\draw (0, 0) node[op amp](opamp){};
			\draw (opamp.+) -- +(0, -1) node[ground](gnd1){};
			\draw (opamp.out) -- +(1, 0) node[ocirc, label=$V_\text{out}$]{};
			\draw (opamp.-) node[circ]{} to[generic, invert, l_=$Z_\text{in}$] +(-3, 0) node(node1){} to[voltage source, l=$V_\text{in}{=}Z_\text{in}I_\text{in}$] (node1 |- gnd1) node[ground]{};
			\draw (opamp.-) |- (-1, 2) to[generic, l=$Z_f$] (1,2) -| (opamp.out);
		\end{circuitikz}
		\caption{Inverting amplifier.}
	\end{subfigure}
	\caption{Equivalence between transimpedance and inverting amplifier using source transformation.}\label{fig:equivalence_transimpedance_inverting}
\end{figure}
\Cref{fig:equivalence_transimpedance_inverting} shows the source transformation applied to transimpedance and inverting amplifier circuits.
Given a current source $I_\text{in}$ with parallel impedance $Z_\text{in}$ the equivalent voltage source has value $V_\text{in}=Z_\text{in}I_\text{in}$ with impedance $Z_\text{in}$ in series.

\subsection{Input offset voltage}

Real operational amplifiers only reduce the voltage difference between the inverting and non-inverting input to a non-zero input offset voltage.
For high-precision operational amplifiers, the input offset voltage is in the range of microvolts.
We can model the input offset voltage as a voltage source at the non-inverting input of an ideal operational amplifier in our transimpedance circuit, as can be seen from \Cref{fig:input_offset_voltage}.
In \Cref{fig:input_offset_voltage} we amended the basic transimpedance circuit of \Cref{fig:basic_transimpedance} by inserting a voltage source with the input offset voltage between the non-inverting input and ground.
\begin{figure}[H]
	\centering
	\begin{circuitikz}
		\draw (0, 0) node[op amp](opamp){};
		\draw (opamp.-) |- (-1, 2) to[generic, l=$Z_f$] (1,2) -| (opamp.out);
		\draw (opamp.out) -- +(1, 0) node[ocirc, label=$V_\text{out}$]{};
		\draw (opamp.+) -- ++(0, -.5) node(node1){} to[voltage source, l=$V_\text{offset}$] ++(0, -2) node[ground](gnd1){};
		\draw (opamp.-) node[circ]{} -- +(-2, 0) node(node2){} -- (node2 |- node1) to[current source, invert, l=$I_\text{in}$] (node2 |- gnd1) node[ground]{};
	\end{circuitikz}
	\caption{Input offset voltage in the transimpedance amplifier.}\label{fig:input_offset_voltage}
\end{figure}
In order to estimate the output offset $V_\text{out}$ caused by the input offset voltage $V_\text{offset}$, we need to use the picture of the inverting amplifier.
For the inverting amplifier, the output voltage follows,
\begin{equation}
	V_\text{out}=-\frac{R_f}{R_\text{in}}V_\text{in},
\end{equation}
wherein $R_\text{in}$ is the resistance parallel to the current source $I_\text{in}$ in the picture of the transimpedance amplifier.
In the case of the position-sensitive detector, the input resistance is of order \SI{10}{\kilo\ohm}.
Using a feedback resistance of $R_f=\SI{100}{\kilo\ohm}$ we find that the input offset voltage experiences a gain of 10.

One approach to compensate for the input offset voltage as described is depicted in \Cref{fig:offset_voltage_compensation}, see also Ref~\cite[p.~54]{Jung05}.
A potentiometer with maximum resistance Rp connects the positive and negative supply voltage.
The wiper of the potentiometer is connected with a first resistor R1 to a node.
The node is connected with a second resistor R2 and an optional bypass capacitor to ground.
Finally, the equivalent input offset voltage source connects the node with the non-inverting input of the operational amplifier.
\begin{figure}[H]
	\centering
	\begin{circuitikz}
		\draw (0, 3) node[ocirc, label=$+V_\text{supply}$]{} to[potentiometer, n=poti, l_=$R_p$] ++(0, -3) node[ocirc, label=$-V_\text{supply}$, rotate=180]{};
		\draw (poti.wiper) to[resistor, l=$R_1$] ++(2, 0) node(node1)[circ]{} to[resistor, l=$R_2$] ++(0, -2) node(gnd)[ground]{};
		\draw (node1) to[voltage source, l_=$V_\text{offset}$, invert] ++(0, 2) node[ocirc, label=$V_+$]{};
		\draw (node1) -- ++(2, 0) to[capacitor, l=$C_1$] ++(0, -2) node[ground]{};
	\end{circuitikz}
	\caption{Input offset voltage compensation using adjustable potentiometer.}\label{fig:offset_voltage_compensation}
\end{figure}
Let $0\leq x\geq1$ be the position of the potentiometer.
For $x=1/2$ the resistance of the potentiometer is $R_p/2$ and there is no offset compensation.
For $x<1/2$ the input offset compensation is negative to compensate for a positive input offset voltage.
For $x>1/2$ the input offset compensation is positive to compensate for a negative input offset voltage.
The maximum input offset compensation is obtained for $x=0$ and $x=1$.
Using circuit analysis techniques, we obtain,
\begin{equation}
	V_c(x)=\frac{R_2V_\text{supply}(1-2x)}{R_1+R_2-R_p(1-x)x},
\end{equation}
as an analytical expression for the input offset voltage compensation measured between the node and ground.
The maximum absolute value of the compensation voltage follows to be,
\begin{equation}
	V_c(0)=V_c(1)=\pm\frac{R_2V_\text{supply}}{R_1+R_2}.
\end{equation}
Given a maximum input offset voltage of \SI{100}{\micro\volt} and a supply voltage of \SI{15}{\volt}, we find approximate resistor values $R_1=\SI{3}{\mega\ohm}$ and $R_2=\SI{2}{\ohm}$.
The resistor value of the potentiometer $R_p$ should be chosen sufficiently large such to limit the current.
For example, a potentiometer resistance of $R_p=\SI{15}{\kilo\ohm}$ would limit the current to \SI{2}{\milli\ampere} with a heat dissipation of \SI{60}{\milli\watt}.
In practice, one should aim for a slightly higher maximum compensation voltage in order to handle resistor mismatches.

That said, there are some practical arguments against the use of the described input offset voltage compensation.
The first argument is that the low resistance of $R_2$ acts as a dominant source for thermal current noise density of about \SI{100}{\pico\ampere\per\sqrt\hertz}.
As this current noise contributes to the input of the transimpedance amplifier it gets amplified by the feedback impedance $Z_f$, yielding up to \SI{100}{\micro\volt\per\sqrt\hertz} in voltage noise density, which --- depending on the bandwidth --- surpasses the actual input offset voltage to compensate.
The second argument is that high-precision potentiometers with large dynamic range get very large and difficult to accommodate on a printed circuit board.

\subsection{Input bias current}

% TODO: mention that I+, I- are difficult to measure and therefore the datasheets report Ibias and Ioffset

\begin{figure}[H]
	\begin{subfigure}[t]{.5\textwidth}
		\centering
		\begin{circuitikz}
			\draw (0, 0) node[op amp, scale=1.2](opamp){};
			\draw (opamp.out) -- +(0.5, 0) node[ocirc, label=$V_\text{out}$]{};
			\draw (opamp.+) -- +(-0.5, 0) node(node1a)[circ]{} to[current source, l=$I_\text{offset}/2$] (node1a |- opamp.-) node(node1b)[circ]{} -- (opamp.-);
			\draw (node1a) -- ++(-1, 0) node(node2a)[circ]{} -- ++(0, -0.5) to[current source, l=$I_\text{bias}$] ++(0, -1) node[ground]{};
			\draw (node1b) -- ++(-1, 0) node(node2b)[circ]{} -- ++(0, +0.5) to[current source, l_=$I_\text{bias}$] ++(0, 1) node[ground, rotate=180]{};
			\draw (node2a) to[short, i<=$i_+$] +(-1.5, 0) node[ocirc, label=$V_+$]{};
			\draw (node2b) to[short, i<_=$i_-$] +(-1.5, 0) node[ocirc, label=$V_-$]{};
		\end{circuitikz}
		\caption{Equivalent current sources as reported in the datasheet.}
	\end{subfigure}
	\begin{subfigure}[t]{.5\textwidth}
		\centering
		\begin{circuitikz}
			\draw (0, 0) node[op amp, scale=1.2](opamp){};
			\draw (opamp.out) -- +(0.5, 0) node[ocirc, label=$V_\text{out}$]{};
			\draw (opamp.+)-- ++(-0.5, 0) node(node2a)[circ]{} -- ++(0, -0.5) to[current source, l=$I_-$] ++(0, -1) node[ground]{};
			\draw (opamp.-) -- ++(-0.5, 0) node(node2b)[circ]{} -- ++(0, +0.5) to[current source, l_=$I_+$] ++(0, 1) node[ground, rotate=180]{};
			\draw (node2a) to[short, i<=$i_+$] +(-1.5, 0) node[ocirc, label=$V_+$]{};
			\draw (node2b) to[short, i<_=$i_-$] +(-1.5, 0) node[ocirc, label=$V_-$]{};
		\end{circuitikz}
		\caption{Alternative equivalent current sources.}
	\end{subfigure}
	\caption{Non-zero input current from the operational amplifier.}
\end{figure}

\begin{align}
	I_+=I_\text{bias}+\frac{1}{2}I_\text{offset} &&
	I_\text{offset}=I_+-I_- \\
	I_-=I_\text{bias}-\frac{1}{2}I_\text{offset} &&
	I_\text{bias}=\frac{I_++I_-}{2}
\end{align}

\begin{figure}[H]
	\centering
	\begin{circuitikz}
		\draw (0, 0) node[op amp](opamp){};
		\draw (opamp.out) -- +(.5, 0) node[ocirc, label=$V_\text{out}$]{};
		\draw (opamp.-) to[resistor, l_=$R_\text{in}$] +(-3, 0) node(node1){} to[voltage source, l=$V_\text{in}$] (node1 |- +0, -3) node[ground](gnd){};
		\draw (opamp.-) |- (-1, 2) to[resistor, l=$R_f$] (1,2) -| (opamp.out);
		\draw (opamp.+) -- ++(0, -1) node(node2)[circ]{} to[resistor, l_=$R_c$] (opamp.+ |- gnd) node[ground]{};
		\draw (node2) -- ++(1, 0) node(node3){} to[capacitor, l=$C_c$] (node3 |- gnd) node[ground]{};
	\end{circuitikz}
	\caption{Input current offset compensation.}
\end{figure}

\cite[p.~57]{Jung05}
\cite[p.~25]{Graeme96}

\begin{equation}
	R_c=\frac{R_\text{in}R_f}{R_\text{in}+R_f}
\end{equation}

% TODO: mention that this only works for well-matched bias currents (Ioffset < Ibias)
% TODO: give noise argument why this introduces more error

\subsection{Noise}

\subsection{Stability}

\cite[p.~693]{Hobbs11}
\cite[p.~183]{Kay12}
\cite[Ch.~5]{Carter17}
\cite[Ch.~3]{Graeme96}