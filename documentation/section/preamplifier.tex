\section{Preamplifier}

The photocurrents created by our detector are in the range of microampere where they are vulnerable to noise.
Using a preamplifier, we can increase the amplitude of the signal for an improved signal-to-noise ratio.
The typical photocurrent preamplifier is based on the transimpedance (current-to-voltage) amplifier design using a voltage-feedback operational amplifier.
Converting the current to a voltage signal has the benefit that the voltage signal can be easily visualized with an oscilloscope.
Furthermore, the voltage-feedback operational amplifier design appears to be more common than the current-feedback operational amplifier, as manufacturers offer much more choice and they are more prominent in the literature.
That said, current-feedback operational amplifiers are reported to be a viable solution for high-speed and high-bandwidth applications, see Ref.~\cite[p.~110]{Jung05} for an overview of the benefits of current-feedback amplifiers and Ref.~\cite[Ch.~9]{Carter17} for a comparison with voltage-feedback amplifiers.

In the following, we will always refer to the voltage-feedback operational amplifier if not noted otherwise.
Moreover, we expect the reader to be familiar with the foundations of the operational amplifier.
Well-written introductions to this topic can be found in \cite[Ch.~1]{Jung05} and \cite[Ch.~3]{Carter17}.

% TODO: mention finite loop gain error (Jung, p. 13)
% TODO: table with op amp parameters specified in the datasheet and their relevance for our application
% TODO: noise model (Jung, p. 80)

\subsection{Basic design}

The basic design of a transimpedance amplifier is illustrated in~\Cref{fig:basic_transimpedance}.
Ignoring imperfections of the photodiode, we can represent the photodiode as a current source.
The non-inverting input of the operational amplifier is connected to ground.
The inverting input is coupled with the output of the operational amplifier via a feedback impedance $Z_f$.
\begin{figure}[H]
	\centering
	\begin{circuitikz}
		\draw (0, 0) node[op amp](opamp){};
		\draw (opamp.+) -- +(0, -1) node[ground](gnd1){};
		\draw (opamp.out) -- +(1, 0) node[ocirc, label=$V_\text{out}$]{};
		\draw (opamp.-) node[circ]{} -- +(-2, 0) node(node1){} to[current source, invert, l=$I_\text{in}$] (node1 |- gnd1) node[ground]{};
		\draw (opamp.-) |- (-1, 2) to[generic, l=$Z_f$] (1,2) -| (opamp.out);
	\end{circuitikz}
	\caption{Basic transimpedance amplifier using voltage feedback operational amplifier.}\label{fig:basic_transimpedance}
\end{figure}
The ideal operation amplifier has zero input current, thus, in the node of the inverting input of the operational amplifier Kirchhoff's law states that the current going through the feedback impedance has to cancel the current of the current source $I_\text{in}$.
The current through the feedback impedance $Z_f$ can be expressed in terms of the feedback impedance $Z_f$ and the output voltage $V_\text{out}$ of the operational amplifier through the use of Ohm's law, yielding,
\begin{equation}
	V_\text{out}=-Z_fI_\text{in}
	\label{eq:transimpedance}.
\end{equation}
Given a maximum current $I_\text{in}$ and a desired maximum output voltage $V_\text{out}$, \Cref{eq:transimpedance} determines the feedback resistance.
Limitations arise for real operational amplifiers where the output voltage is limited to be below the supply voltage of the operational amplifier.


Aside from photodiode applications, it is more common to find the inverting (voltage-to-voltage) operational amplifier in the literature.
By using the source transformation on the current source with a parallel impedance in the transimpedance amplifier circuit, we can recover the inverting operational amplifier circuit, and thereby easily use results obtained for this design.
\begin{figure}[H]
	\begin{subfigure}[t]{.5\textwidth}
		\centering
		\begin{circuitikz}
			\draw (0, 0) node[op amp](opamp){};
			\draw (opamp.+) -- +(0, -1) node[ground](gnd1){};
			\draw (opamp.out) -- +(1, 0) node[ocirc, label=$V_\text{out}$]{};
			\draw (opamp.-) node[circ]{} -- +(-3, 0) node(node1){} to[current source, invert, l=$I_\text{in}$] (node1 |- gnd1) node[ground]{};
			\draw (opamp.-) node[circ]{} -- +(-1.5, 0) node(node1){} to[generic, invert, l=$Z_\text{in}$] (node1 |- gnd1) node[ground]{};
			\draw (opamp.-) |- (-1, 2) to[generic, l=$Z_f$] (1,2) -| (opamp.out);
		\end{circuitikz}
		\caption{Transimpedance amplifier.}
	\end{subfigure}
	\begin{subfigure}[t]{.5\textwidth}
		\centering
		\begin{circuitikz}
			\draw (0, 0) node[op amp](opamp){};
			\draw (opamp.+) -- +(0, -1) node[ground](gnd1){};
			\draw (opamp.out) -- +(1, 0) node[ocirc, label=$V_\text{out}$]{};
			\draw (opamp.-) node[circ]{} to[generic, invert, l_=$Z_\text{in}$] +(-3, 0) node(node1){} to[voltage source, invert, l=$V_\text{in}{=}Z_\text{in}I_\text{in}$] (node1 |- gnd1) node[ground]{};
			\draw (opamp.-) |- (-1, 2) to[generic, l=$Z_f$] (1,2) -| (opamp.out);
		\end{circuitikz}
		\caption{Inverting amplifier.}
	\end{subfigure}
	\caption{Equivalence between transimpedance and inverting amplifier using source transformation.}\label{fig:equivalence_transimpedance_inverting}
\end{figure}
\Cref{fig:equivalence_transimpedance_inverting} shows the source transformation applied to transimpedance and inverting amplifier circuits.
Given a current source $I_\text{in}$ with parallel impedance $Z_\text{in}$ the equivalent voltage source has value $V_\text{in}=Z_\text{in}I_\text{in}$ with impedance $Z_\text{in}$ in series.

\subsection{Offset}

\subsubsection{Input offset voltage}

\cite[p.~54]{Jung05}

\begin{figure}[H]
	\centering
	\begin{circuitikz}
		\draw (0, 0) node[op amp](opamp){};
		\draw (opamp.out) -- +(.5, 0) node[ocirc, label=$V_\text{out}$]{};
		\draw (opamp.-) to[resistor, l_=$R_\text{in}$] +(-3, 0) node(node1){} to[voltage source, l=$V_\text{in}$] (node1 |- +0, -3) node[ground](gnd){};
		\draw (opamp.-) |- (-1, 2) to[resistor, l=$R_f$] (1,2) -| (opamp.out);
		\draw (opamp.+) -- ++(0, -1) node(node2)[circ]{};
		\draw (-2, -0.5) to[potentiometer, n=pot, l_=$R_p$] (-2, -2.5);
		\draw (node2) -- ++(2, 0) node(node3){} to[resistor, l_=$R_c$] (node3 |- gnd) node[ground]{};
		\draw (node2) -- (pot.wiper);
		\draw (-2, -0.5) node[ocirc, label=$+V_s$]{};
		\draw (-2, -2.5) node[ocirc, label=$-V_s$, rotate=180]{};
	\end{circuitikz}
	\caption{Input current offset compensation.}
\end{figure}

\subsubsection{Input bias current}

% TODO: mention that I+, I- are difficult to measure and therefore the datasheets report Ibias and Ioffset

\begin{figure}[H]
	\begin{subfigure}[t]{.5\textwidth}
		\centering
		\begin{circuitikz}
			\draw (0, 0) node[op amp, scale=1.2](opamp){};
			\draw (opamp.out) -- +(0.5, 0) node[ocirc, label=$V_\text{out}$]{};
			\draw (opamp.+) -- +(-0.5, 0) node(node1a)[circ]{} to[current source, l=$I_\text{offset}/2$] (node1a |- opamp.-) node(node1b)[circ]{} -- (opamp.-);
			\draw (node1a) -- ++(-1, 0) node(node2a)[circ]{} -- ++(0, -0.5) to[current source, l=$I_\text{bias}$] ++(0, -1) node[ground]{};
			\draw (node1b) -- ++(-1, 0) node(node2b)[circ]{} -- ++(0, +0.5) to[current source, l_=$I_\text{bias}$] ++(0, 1) node[ground, rotate=180]{};
			\draw (node2a) to[short, i<=$i_+$] +(-1.5, 0) node[ocirc, label=$V_+$]{};
			\draw (node2b) to[short, i<_=$i_-$] +(-1.5, 0) node[ocirc, label=$V_-$]{};
		\end{circuitikz}
		\caption{Equivalent current sources as reported in the datasheet.}
	\end{subfigure}
	\begin{subfigure}[t]{.5\textwidth}
		\centering
		\begin{circuitikz}
			\draw (0, 0) node[op amp, scale=1.2](opamp){};
			\draw (opamp.out) -- +(0.5, 0) node[ocirc, label=$V_\text{out}$]{};
			\draw (opamp.+)-- ++(-0.5, 0) node(node2a)[circ]{} -- ++(0, -0.5) to[current source, l=$I_-$] ++(0, -1) node[ground]{};
			\draw (opamp.-) -- ++(-0.5, 0) node(node2b)[circ]{} -- ++(0, +0.5) to[current source, l_=$I_+$] ++(0, 1) node[ground, rotate=180]{};
			\draw (node2a) to[short, i<=$i_+$] +(-1.5, 0) node[ocirc, label=$V_+$]{};
			\draw (node2b) to[short, i<_=$i_-$] +(-1.5, 0) node[ocirc, label=$V_-$]{};
		\end{circuitikz}
		\caption{Alternative equivalent current sources.}
	\end{subfigure}
	\caption{Non-zero input current from the operational amplifier.}
\end{figure}

\begin{align}
	I_+=I_\text{bias}+\frac{1}{2}I_\text{offset} &&
	I_\text{offset}=I_+-I_- \\
	I_-=I_\text{bias}-\frac{1}{2}I_\text{offset} &&
	I_\text{bias}=\frac{I_++I_-}{2}
\end{align}

\begin{figure}[H]
	\centering
	\begin{circuitikz}
		\draw (0, 0) node[op amp](opamp){};
		\draw (opamp.out) -- +(.5, 0) node[ocirc, label=$V_\text{out}$]{};
		\draw (opamp.-) to[resistor, l_=$R_\text{in}$] +(-3, 0) node(node1){} to[voltage source, l=$V_\text{in}$] (node1 |- +0, -3) node[ground](gnd){};
		\draw (opamp.-) |- (-1, 2) to[resistor, l=$R_f$] (1,2) -| (opamp.out);
		\draw (opamp.+) -- ++(0, -1) node(node2)[circ]{} to[resistor, l_=$R_c$] (opamp.+ |- gnd) node[ground]{};
		\draw (node2) -- ++(1, 0) node(node3){} to[capacitor, l=$C_c$] (node3 |- gnd) node[ground]{};
	\end{circuitikz}
	\caption{Input current offset compensation.}
\end{figure}

\cite[p.~57]{Jung05}
\cite[p.~25]{Graeme96}

\begin{equation}
	R_c=\frac{R_\text{in}R_f}{R_\text{in}+R_f}
\end{equation}

% TODO: mention that this only works for well-matched bias currents (Ioffset < Ibias)
% TODO: give noise argument why this introduces more error

\subsection{Noise}

\subsection{Stability}

\cite[p.~693]{Hobbs11}
\cite[p.~183]{Kay12}
\cite[Ch.~5]{Carter17}
\cite[Ch.~3]{Graeme96}