\section{Position-sensitive detector}

The \gls{psd} constitutes the heart of the position-sensitive device.
Its characteristics give the upper bound of the performance of our device.
In the following section we will give an overview of the available methods for optical position measurement in order to motivate the selection of a tetra-lateral \gls{psd} photodiode.
The arguments given are an excerpt of Ref.~\cite{Noorlag74}.
To the end of this section we describe how the \gls{psd} can be integrated into the framework of electrical circuit analysis.

\subsection{Detector overview}



\subsection{Operating principle}

Motivate the choice of the \gls{psd} to be an improved tetra-lateral PSD photodiode compared to i.e.\ quadrant diodes and explain the working principle~\cite{Noorlag74}.

For high speed applications one usually connects the cathode to a reverse bias voltage.
However, in our case we are interested in a high spatial resolution. In this case ...ref... the \gls{psd} cathode should be connected to ground.

\subsection{Detector selection}

We have chosen the \gls{s5990} as the position-sensitive detector.
The parameters of the \gls{s5990} are reported in \Cref{tab:s5990}.
\begin{table}[H]
	\centering
	\begin{tabular}{lllll}
		\toprule
			\multirow{2}[3]{*}{Parameter} &
			\multirow{2}[3]{*}{Symbol} &
			\multicolumn{2}{c}{Values} &
			\multirow{2}[3]{*}{Unit} \\
			\cmidrule(lr){3-4} & & Typical & Maximum & \\
		\midrule
		Dark current & $I_\text{dark}$ & \num{0.5} & \num{10} & \si{\nano\ampere}\\
		Interelectrode resistance & $R_\text{ie}$ & \num{7} & \num{15} & \si{\kilo\ohm}\\
		Terminal capacitance & $C_t$ & \num{150} & \num{300} & \si{\pico\farad}\\
		Position resolution & $\Delta x$ & \num{0.7} & & \si{\micro\meter}\\
		\bottomrule	
	\end{tabular}
	\caption{Important parameters of the \gls{s5990} extracted from the datasheet~\cite{HamamatsuPSD}.}\label{tab:s5990}
\end{table}

\subsection{Equivalent circuit}

The output terminals of the \gls{psd} can be modelled as a current source with internal resistance and capacitance~\cite{HamamatsuPSD}.
\begin{figure}[H]
	\centering
	\begin{circuitikz}
		\draw (0, 0)
			node[ground] {}
			to[american current source, l=$I_\text{photo}$] (0, 4)
			to[short, -*] (2, 4)
			to[short, -*] (4, 4)
			to[short] (6, 4)
			to[short] (6, 0)
			to[short, -*] (4, 0)
			to[short, -*] (2, 0)
			to[short] (0, 0);
		\draw (2, 0)
			to[american current source, l=$I_\text{dark}$] (2, 4);
		\draw (4, 0)
			to[european resistor, l=$R_\text{ie}$] (4, 4);
		\draw (6, 0)
			to[capacitor, l=$C_\text{t}$] (6, 4);
		\draw (6, 4)
			to[short, -o] (6, 5);
	\end{circuitikz}
	\caption{Equivalent circuit of one of the \gls{psd} output terminals according to~\cite{HamamatsuPSD}.}\label{fig:circuit:psd}
\end{figure}