\section{Transimpedance amplifier}

% need motivation why current-to-voltage (transimpedance) design

% explain amplification factor
% mention feedback tee network as workaround to high Rf

\begin{figure}[H]
	\centering
	\begin{circuitikz}
		\draw
			(0, 0)
				node[op amp] (opamp) {}
			(opamp.-)
				to[short] (-1, 0.5)
				to[american current source, invert, label=I\textsubscript{in}] (-4, 0.5)
				to[short] (-4, -1.5)
				node[ground] {}
			(opamp.-)
				to[short, *-] ++(0, 1.5)
				coordinate (left)
				to[resistor, label=R\textsubscript{f}] (left -| opamp.out)
				to[short, -*] (opamp.out)
			(opamp.+)
				to[short] +(0, -1)
				node[ground] {}
			(opamp.out)
				to[short] (3, 0)
				node[ocirc, label=V\textsubscript{out}] {}
	;
	\end{circuitikz}
	\caption{Simple transimpedance amplifier circuit.}\label{fig:circuit_transimpedance_amplifier}
\end{figure}

\subsection{Offset compensation}

% ideal to real op-amp
\begin{figure}[H]
	\centering
	\begin{circuitikz}
		\draw
			(0, 0)
				node[op amp] (opamp) {}
			(opamp.-)
				to[short] (-1, 0.5)
				to[american current source, invert, label=I\textsubscript{in}] (-4, 0.5)
				to[short] (-4, -1.5)
				node[ocirc, label=I\textsubscript{-}] {}
			(opamp.+)
				to[short] +(0, -1)
				node[ocirc, label=I\textsubscript{+}] {}
			(opamp.out)
				to[short] (3, 0)
				node[ocirc, label=V\textsubscript{out}] {}
	;
	\end{circuitikz}
	\caption{Equivalent circuit of an operational amplifier comprising input bias and input offset current, and input offset voltage.}\label{fig:circuit_nonideal_opamp}
\end{figure}

\subsubsection{Input bias current}

% can compensate by impedance matching

\subsubsection{Input offset current}

% cannot compensate input offset current

\subsubsection{Input offset voltage}

% can compensate through voltage injection

\subsection{Frequency response}

