\section{Transimpedance amplifier}

% TODO: motivate choice for current-to-voltage (transimpedance) design
% TODO: mention feedback tee network for high feedback factor

\begin{figure}[H]
	\centering
	\begin{circuitikz}
		\draw (0, 0) node[op amp](opamp){};
		\draw (opamp.+) -- +(0, -1) node[ground](gnd1){};
		\draw (opamp.out) -- (3, 0) node[ocirc, label=$V_\text{out}$]{};
		\draw (opamp.-) to[current source, invert, l=$I_\text{in}$] +(-3, 0) node(node1){} -- (node1 |- gnd1) node[ground]{};
		\draw (opamp.-) |- (-1, 2) to[resistor, l=$R_f$] (1,2) -| (opamp.out);
	\end{circuitikz}
	\caption{Simple transimpedance amplifier circuit.}
\end{figure}

\begin{equation}
	V_\text{out}=R_fI_\text{in}
\end{equation}

\begin{figure}[H]
	\begin{subfigure}[t]{.5\textwidth}
		\centering
		\begin{circuitikz}
			\draw (0, 0) node[op amp](opamp){};
			\draw (opamp.+) -- +(0, -1) node[ground](gnd1){};
			\draw (opamp.out) -- +(.5, 0) node[ocirc, label=$V_\text{out}$]{};
			\draw (opamp.-) -- +(-3, 0) node(node1){} to[current source, invert, l=$I_\text{in}$] (node1 |- gnd1) node[ground]{};
			\draw (node1) -- +(1.5, 0) node(node2){} to[resistor, l=$R_\text{in}$] (node2 |- gnd1) node[ground]{};
			\draw (opamp.-) |- (-1, 2) to[resistor, l=$R_f$] (1,2) -| (opamp.out);
		\end{circuitikz}
		\caption{Transimpedance amplifier.}
	\end{subfigure}
	\begin{subfigure}[t]{.5\textwidth}
		\centering
		\begin{circuitikz}
			\draw (0, 0) node[op amp](opamp){};
			\draw (opamp.+) -- +(0, -1) node[ground](gnd1){};
			\draw (opamp.out) -- +(.5, 0) node[ocirc, label=$V_\text{out}$]{};
			\draw (opamp.-) to[resistor, l_=$R_\text{in}$] +(-3, 0) node(node1){} to[voltage source, l=$V_\text{in}{=}R_\text{in}I_\text{in}$] (node1 |- gnd1) node[ground]{};
			\draw (opamp.-) |- (-1, 2) to[resistor, l=$R_f$] (1,2) -| (opamp.out);
		\end{circuitikz}
		\caption{Inverting amplifier.}
	\end{subfigure}
	\caption{Equivalence between transimpedance and inverting amplifier using source transformation.}
\end{figure}

\subsection{Noise}

% TODO: add shot noise
% TOOD: add thermal noise

\subsection{Offset}

% TODO: mention photodiode leakage current (Graemer p. 25)
% TODO: compensation circuit (Jung p. 57)

\subsubsection{Input current}

% TODO: mention that I+, I- are difficult to measure and therefore the datasheets report Ibias and Ioffset

\begin{figure}[H]
	\begin{subfigure}[t]{.5\textwidth}
		\centering
		\begin{circuitikz}
			\draw (0, 0) node[op amp, scale=1.2](opamp){};
			\draw (opamp.out) -- +(0.5, 0) node[ocirc, label=$V_\text{out}$]{};
			\draw (opamp.+) -- +(-0.5, 0) node(node1a)[circ]{} to[current source, l=$I_\text{offset}/2$] (node1a |- opamp.-) node(node1b)[circ]{} -- (opamp.-);
			\draw (node1a) -- ++(-1, 0) node(node2a)[circ]{} -- ++(0, -0.5) to[current source, l=$I_\text{bias}$] ++(0, -1) node[ground]{};
			\draw (node1b) -- ++(-1, 0) node(node2b)[circ]{} -- ++(0, +0.5) to[current source, l_=$I_\text{bias}$] ++(0, 1) node[ground, rotate=180]{};
			\draw (node2a) to[short, i<=$i_+$] +(-1.5, 0) node[ocirc, label=$V_+$]{};
			\draw (node2b) to[short, i<_=$i_-$] +(-1.5, 0) node[ocirc, label=$V_-$]{};
		\end{circuitikz}
		\caption{Equivalent current sources as reported in the datasheet.}
	\end{subfigure}
	\begin{subfigure}[t]{.5\textwidth}
		\centering
		\begin{circuitikz}
			\draw (0, 0) node[op amp, scale=1.2](opamp){};
			\draw (opamp.out) -- +(0.5, 0) node[ocirc, label=$V_\text{out}$]{};
			\draw (opamp.+)-- ++(-0.5, 0) node(node2a)[circ]{} -- ++(0, -0.5) to[current source, l=$I_-$] ++(0, -1) node[ground]{};
			\draw (opamp.-) -- ++(-0.5, 0) node(node2b)[circ]{} -- ++(0, +0.5) to[current source, l_=$I_+$] ++(0, 1) node[ground, rotate=180]{};
			\draw (node2a) to[short, i<=$i_+$] +(-1.5, 0) node[ocirc, label=$V_+$]{};
			\draw (node2b) to[short, i<_=$i_-$] +(-1.5, 0) node[ocirc, label=$V_-$]{};
		\end{circuitikz}
		\caption{Alternative equivalent current sources.}
	\end{subfigure}
	\caption{Non-zero input current from the operational amplifier.}
\end{figure}

\begin{align}
	I_+=I_\text{bias}+\frac{1}{2}I_\text{offset} &&
	I_\text{offset}=I_+-I_- \\
	I_-=I_\text{bias}-\frac{1}{2}I_\text{offset} &&
	I_\text{bias}=\frac{I_++I_-}{2}
\end{align}

\subsubsection{Input offset voltage}

\subsubsection{Offset compensation}

\subsection{Bandwidth}

