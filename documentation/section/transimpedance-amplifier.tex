\section{Transimpedance amplifier}

% need motivation why current-to-voltage (transimpedance) design

% explain amplification factor
% mention feedback tee network as workaround to high Rf

\begin{figure}[H]
	\centering
	\begin{circuitikz}
		\draw (0, 0) node[op amp](opamp){};
		\draw (opamp.+) -- +(0, -1) node[ground](gnd1){};
		\draw (opamp.out) -- (3, 0) node[ocirc, label=V\textsubscript{out}]{};
		\draw (opamp.-) to[current source, invert, label=I\textsubscript{in}] +(-3, 0) node(node1){} -- (node1 |- gnd1) node[ground]{};
		\draw (opamp.-) |- (-1, 2) to[resistor, label=R\textsubscript{f}] (1,2) -| (opamp.out);
	\end{circuitikz}
	\caption{Basic transimpedance amplifier circuit.}\label{fig:circuit_transimpedance_amplifier}
\end{figure}

\subsection{Offset compensation}

% ideal to real op-amp
\begin{figure}[H]
	\centering
	\begin{circuitikz}
		\draw (0, 0) node[op amp, scale=1.5](opamp){};
		\draw (opamp.out) -- +(1, 0) node[ocirc, label=V\textsubscript{out}]{};
		\draw (opamp.+) to[voltage source, label=V\textsubscript{offset}] ++(-1.5, 0) -- ++(-0.5, 0) node(node1)[circ]{};
		\draw (node1) to[current source, label=I\textsubscript{offset}/2] (node1 |- opamp.-) node(node2)[circ]{} -- (opamp.-);
		\draw (node2) -- ++(-4.5, 0) node(node3)[circ]{} -- (node3 |- opamp.+) to[current source, label=I\textsubscript{bias}]  ++(0, -2) node[ground]{};
		\draw(node1) -- ++(-2.5, 0) node(node4)[circ]{} to[current source, label=I\textsubscript{bias}] ++(0, -2) node[ground]{};
		\draw (node3) to[short, f<_=i\textsubscript{-}] +(-2, 0) node[ocirc, label=V\textsubscript{-}]{};
		\draw (node4) -- +(-2, 0) to[short, f<_=i\textsubscript{+}] +(-4, 0) node[ocirc, label=V\textsubscript{+}]{};
	\end{circuitikz}
	\caption{Equivalent circuit of an operational amplifier comprising input bias current, input offset current and input offset voltage.}\label{fig:circuit_opamp_offset_bias}
\end{figure}

\subsubsection{Input bias current}

% can compensate by impedance matching

\subsubsection{Input offset current}

% cannot compensate input offset current

\subsubsection{Input offset voltage}

% can compensate through voltage injection

\subsection{Frequency response}

