\section{Transimpedance amplifier}

% need motivation why current-to-voltage (transimpedance) design

% explain amplification factor
% mention feedback tee network as workaround to high Rf

\begin{figure}[H]
	\centering
	\begin{circuitikz}
		\draw (0, 0) node[op amp](opamp){};
		\draw (opamp.+) -- +(0, -1) node[ground](gnd1){};
		\draw (opamp.out) -- (3, 0) node[ocirc, label=$V_\text{out}$]{};
		\draw (opamp.-) to[current source, invert, l=$I_\text{in}$] +(-3, 0) node(node1){} -- (node1 |- gnd1) node[ground]{};
		\draw (opamp.-) |- (-1, 2) to[resistor, l=$R_f$] (1,2) -| (opamp.out);
	\end{circuitikz}
	\caption{Simple transimpedance amplifier circuit.}
\end{figure}

\begin{equation}
	V_\text{out}=R_fI\text{in}
\end{equation}

\begin{figure}[H]
	\begin{subfigure}[t]{.5\textwidth}
		\centering
		\begin{circuitikz}
			\draw (0, 0) node[op amp](opamp){};
			\draw (opamp.+) -- +(0, -1) node[ground](gnd1){};
			\draw (opamp.out) -- +(.5, 0) node[ocirc, label=$V_\text{out}$]{};
			\draw (opamp.-) -- +(-3, 0) node(node1){} to[current source, invert, l=$I_\text{in}$] (node1 |- gnd1) node[ground]{};
			\draw (node1) -- +(1.5, 0) node(node2){} to[resistor, l=$R_\text{in}$] (node2 |- gnd1) node[ground]{};
			\draw (opamp.-) |- (-1, 2) to[resistor, l=$R_f$] (1,2) -| (opamp.out);
		\end{circuitikz}
		\caption{Transimpedance amplifier.}
	\end{subfigure}
	\begin{subfigure}[t]{.5\textwidth}
		\centering
		\begin{circuitikz}
			\draw (0, 0) node[op amp](opamp){};
			\draw (opamp.+) -- +(0, -1) node[ground](gnd1){};
			\draw (opamp.out) -- +(.5, 0) node[ocirc, label=$V_\text{out}$]{};
			\draw (opamp.-) to[resistor, l_=$R_\text{in}$] +(-3, 0) node(node1){} to[voltage source, l=$V_\text{in}{=}R_\text{in}I_\text{in}$] (node1 |- gnd1) node[ground]{};
			\draw (opamp.-) |- (-1, 2) to[resistor, l=$R_f$] (1,2) -| (opamp.out);
		\end{circuitikz}
		\caption{Inverting amplifier.}
	\end{subfigure}
	\caption{Equivalence between transimpedance and inverting amplifier using source transformation.}
\end{figure}

\subsection{Offset voltage}

% ideal to real op-amp
\begin{figure}[H]
	\centering
	\begin{circuitikz}
		\draw (0, 0) node[op amp, scale=1.5](opamp){};
		\draw (opamp.out) -- +(1, 0) node[ocirc, label=$V_\text{out}$]{};
		\draw (opamp.+) to[voltage source, label=$V_\text{offset}$] ++(-1.5, 0) -- ++(-0.5, 0) node(node1)[circ]{};
		\draw (node1) to[current source, label=$I_\text{offset}/2$] (node1 |- opamp.-) node(node2)[circ]{} -- (opamp.-);
		\draw (node2) -- ++(-4.5, 0) node(node3)[circ]{} -- (node3 |- opamp.+) to[current source, label=$I_\text{bias}$]  ++(0, -2) node[ground]{};
		\draw(node1) -- ++(-2.5, 0) node(node4)[circ]{} to[current source, label=$I_\text{bias}$] ++(0, -2) node[ground]{};
		\draw (node3) to[short, f<_=$i_-$] +(-2, 0) node[ocirc, label=$V_-$]{};
		\draw (node4) -- +(-2, 0) to[short, f<_=$i_+$] +(-4, 0) node[ocirc, label=$V_+$]{};
	\end{circuitikz}
	\caption{Equivalent circuit of an imperfect operational amplifier comprising input bias current, input offset current and input offset voltage.}\label{fig:circuit_opamp_offset_bias}
\end{figure}

% can compensate by impedance matching

The input offset current $I_\text{offset}$ denotes the difference between the input current of the input terminals,
\begin{equation}
	I_\text{offset}=i_+-i_-.
\end{equation}
In the case of the simple transimpedance amplifier a non-zero input offset current adds an output voltage offset $R_fI_\text{offset}$, see Ref.~\cite[p.~57]{Jung05}.


% cannot compensate input offset current

% can compensate through voltage injection

\subsection{Frequency response}

\subsection{Noise reduction}

