\section{Introduction}

The performance of high-precision optical setups is determined by the magnitude of the variations of some critical parameters.
One of these critical parameters is the beam alignment.
The initial beam alignment is crucial in order to reduce optical aberrations by i.e.\ diffraction.
Once performed temperature fluctuations, mechanical strain or human interaction can alter the beam alignment or render the initial beam alignment obsolete.

Given these circumstances it is therefore an obvious step to integrate a control loop into the optical setup that corrects for errors in the beam alignment.
The control loop comprises a position-sensitive device to gather the current state of the beam alignment as well as a mechanical mirror to compensate for errors. 

In the present document we want to summarise the design process of such a position-sensitive device.

\subsection{Requirements}

\begin{table}[H]
	\centering
	\begin{tabular}{llccc}
		\toprule
			\multirow{2}[3]{*}{Parameter} &
			\multirow{2}[3]{*}{Symbol} &
			\multicolumn{2}{c}{Values} &
			\multirow{2}[3]{*}{Unit} \\
			\cmidrule(lr){3-4} & & Best & Worst & \\
		\midrule
			Spatial resolution & $\Delta x$ & \num{0.7} & & \si{\micro\meter} \\
		\bottomrule
	\end{tabular}
	\caption{Technical requirements of the position-sensitive device.}
\end{table}

\begin{enumerate}
    \item Dual power supply $V_\pm=\pm\SI{15}{\volt}$
    \item Ethernet network interface
    \item Analogue interface
\end{enumerate}

\subsection{Overview}

Components of the device.