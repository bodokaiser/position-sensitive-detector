\section{Introduction}

We define a position-sensitive device as a device that converts the center of mass of an input optical signal into an electrical output.
The present document is meant to summarize the theory required to design such a position-sensitive device.

Position-sensitive devices are used in a wide range of industrial and commercial applications, including displacement sensing and beam alignment, see Ref.~\cite[p.~22]{Maekynen00}.
In our case, we were interested in the application of the position-sensitive device for beam pointing alignment.
Beam pointing refers to the spatial focus of a laser beam and can change through, for instance, thermal and mechanical effects.
Uncompensated changes in beam alignment can quickly degrade the overall performance of an optical system.
It is, therefore, crucial to align the beam pointing to ensure proper operation of the optical system at hand.
Using a position-sensitive device for monitoring the beam pointing as part of a feedback loop, would allow for automatization of this otherwise cumbersome work.

This document is organized as follows.
The first section gives an overview of the design requirements imposed upon our position-sensitive device.
In the second section, we summarize the physical theory behind the position-sensitive detector and discuss its electrical properties.
In the third section, we dive into the theory of operational amplifier as a front-end for the position-sensitive detector.