\documentclass{standalone}

\ifstandalone
	\usepackage{amsmath}
	\usepackage{circuitikz}
\fi


\begin{document}
	\begin{circuitikz}
		\draw (0, 0) node[vcc, rotate=90, label={[label distance=6]:\SI{15}{\volt}}]{} -- ++(1, 0) node(node1a)[circ]{} to[capacitor, l_=C1, a^=\SI{1}{\micro\farad}] ++(0, -4) node(node2a)[circ]{} -- ++(-1, 0) node[ground, rotate=270]{};
		\draw (node1a) -- ++(2, 0) node(node1b)[circ]{} -- ++(0, -4) node(node2b)[circ]{} -- (node2a);
		\draw (node1b) -- ++(2, 0) node(node1c)[circ]{} to[capacitor, l_=C2, a^=\SI{1}{\micro\farad}] ++(0, -4) node(node2c)[circ]{} -- (node2a);
		\draw (node1c) -- ++(2, 0) node(node1d)[circ]{} to[resistor, l_=R1, a^=\SI{100}{\ohm}] ++(0, -2) to[capacitor, l_=C3, a^=\SI{10}{\nano\farad}] ++(0, -2) node(node2d)[circ]{} -- (node2c);
		\draw (node1d) -- ++(2, 0) node(node1e)[circ]{} to[ecapacitor, l_=C4, a^=\SI{22}{\micro\farad}] ++(0, -4) node(node2e)[circ]{} -- (node2c);
		\draw (node1e) -- ++(1, 0) node[vcc, rotate=270, label={[label distance=6]:\SI{10}{\volt}}]{};
		\draw (node2e) -- ++(1, 0) node[ground, rotate=90]{};
		% TODO: use dipchip node and name IC ports
		\begin{scope}[xshift=4cm]
			\node[draw, rectangle, fill=white, minimum width=3cm, minimum height=1.4cm, label=above:U1]{};
			\node at (0, 0) {REF5010};
		\end{scope}
	\end{circuitikz}
\end{document}
