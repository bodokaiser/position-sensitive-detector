\section{Position-sensitive photodiode}

The present section gives an introduction to the semiconductor theory for a position-sensitive photodiode.
We start from the description of a p-n junction and extend the description to the position-sensitive photodiode.
After that, we discuss the evolution of position-sensitive photodiode designs.
Finally, we derive an analytical relation between the photocurrents and the light focus's position on the sensitive area.

\subsection{Transverse photodiodes}

The purpose of the following section is to recall the mechanics of the (transverse) photoeffect observed in an illuminated p-n junction.
Figures and the description thereof are primarily based on Ref.~\cite{Simon13}.
A subtle difference in the depicted figures and the figures of Ref.~\cite{Simon13} is that we exchanged the order of p- and n-type semiconductors.
We found this order to be more intuitive when referring to a p-n junction.
\begin{figure}[H]
	\centering
	\includestandalone[mode=buildnew]{figure/diagram/pn-junction-separated}
	\caption{Separated p- and n-type semiconductor with holes (white) and electrons (black).}\label{fig:pn_junction_separated}
\end{figure}
\Cref{fig:pn_junction_separated} shows a separated p- and n-type semiconductor.
We see an illustration of the p- and n-type semiconductors with their respective mobile charge carriers in the upper half.
The p-type semiconductor has an excess of positive charge carriers (holes), depicted as white circles.
The n-type semiconductor has an excess of negative charge carriers (electrons), depicted as black circles.
The excess charge carriers form due to the implantation of acceptor and donator ions and are indicated as circles with plus and minus sign in \Cref{fig:pn_junction_separated}.
The implanted ions have more or fewer electrons than the atoms of the semiconductor material.
Therefore donating an electron or accepting an electron and effectively forming a hole as an absence of negative charge.
The lower half of \Cref{fig:pn_junction_separated} shows the energy band structure of both p- and n-type semiconductors.
The lower energy band represents the valence band made up of the tightly bound electrons.
The upper energy band represents the conduction band made up of electrons that are are not bound to a single atomic core but shared across the lattice.
Charge carriers in the conduction band can move freely and thereby contribute to the conductivity of the material.
For an undoped (intrinsic) semiconductor, the chemical potential is in the center of the bandgap between the conduction and valence band.
Doping shifts the chemical potential in the p- and n-type semiconductors.
In the p-type semiconductor, acceptor ions can take up electrons from the conduction band, thereby decreasing the chemical potential.
In the n-type semiconductor, donator ions contribute electrons to the conduction band, increasing the chemical potential.
\begin{figure}[H]
	\centering
	\includestandalone[mode=buildnew]{figure/diagram/pn-junction-combined}
	\caption{Combined p- and n-type semiconductor with holes (white) and electrons (black).}\label{fig:pn_junction_combined}
\end{figure}
In \Cref{fig:pn_junction_combined} the p- and n-type semiconductors are brought into contact with each other, forming a p-n junction.
Close to the junction, holes and electrons recombine due to a diffusion process, leaving an electrically charged area.
The electrically charged area creates an electrostatic potential across the junction, as illustrated in the lower part of \Cref{fig:pn_junction_combined}.
We refer to this area as the depletion region.
In \Cref{fig:pn_junction_combined} the depletion region expands between the dashed lines around the junction.
\begin{figure}[H]
	\centering
	\includestandalone[mode=buildnew]{figure/diagram/pn-junction-energy}
	\caption{Energy bands of the p-n junction.}\label{fig:pn_junction}
\end{figure}
The energy band diagram in \Cref{fig:pn_junction} accounts for the shift in energy due to the electrostatic potential.
The chemical potentials of both sides of the junction are now aligned.
The energy required to excite an electron on the p-type side from the valence to the conduction band and the energy needed to excite a hole from the conduction to the valence band are equal to the bandgap of the semiconductor.
Suppose one applies a reverse bias voltage across the p-n junction. In that case, the effective energy gap between conduction and valence band is reduced, the electrostatic potential increases, and the depletion region broadens.
\Cref{fig:pn_junction_reverse} depicts the situation of an applied reverse voltage.
\begin{figure}[H]
	\centering
	\includestandalone[mode=buildnew]{figure/diagram/pn-junction-energy-reverse}
	\caption{Energy band diagram of a reverse biased p-n junction.}\label{fig:pn_junction_reverse}
\end{figure}
The current-voltage characteristic of the p-n junction is described by the Schockley diode equation,
\begin{equation}
	I_\text{diode}=I_\text{sat}(T)\left(e^{eV/k_BT}-1\right)
	\label{eq:diode_current},
\end{equation}
wherein $I_\text{sat}\propto e^{-E_\text{gap}/k_BT}$ is the temperature dependent reverse bias saturation current and $V$ the voltage applied to the p-n junction.
Using the proportionality of the reverse bias saturation current, we can write,
\begin{equation}
	I_\text{diode}\propto e^{\left(eV-E_\text{gap}\right)/k_BT}-e^{-E_\text{gap}/k_BT}\label{eq:diode_current_prop},
\end{equation} 
which discloses the two effects contributing to the diode current.
The left-hand side of the proportionality of \Cref{eq:diode_current_prop} represents the current contribution due to intra-band excitation of charge carriers, whereas the right-hand side represents the current contribution due to inter-band excitation.
\begin{figure}[H]
	\centering
	\includestandalone[mode=buildnew]{figure/plot/diode-current-voltage}
	\caption{Current-voltage characteristics of a p-n junction with different levels of illumination.}\label{fig:pn_junction_iv}
\end{figure}
In \Cref{fig:pn_junction_iv} we see the current-voltage characteristics of the p-n junction under different levels of illumination.
For negative voltages, the p-n junction is operated under reverse bias.
If the reverse bias voltage exceeds the breakdown voltage, the p-n junction starts to conduct.
The reverse saturation current is the amount of current necessary for the breakdown. The curve shifts downwards with increasing illumination.
The separation between the non-illuminated (top curve) and illuminated curves represent the respective photocurrent.

The conversion rate of photons to photoelectrons depends on the type of the bandgap, i.e., direct or indirect, the wavelength $\lambda$ of the photon, and the temperature $T$.
Most photodiodes report a wavelength $\lambda$ dependent responsitivity $R$ which can be used to convert the radiant flux $P$ of the incident light to the generated photocurrent $I_\text{photo}$,
\begin{equation}
	I_\text{photo}=R(\lambda)P
	\label{eq:responsitivity}.
\end{equation}
For silicon-based p-n junctions, the responsitivity $R(\lambda)$ is between \SI{0.2}{\ampere\per\watt} at \SI{400}{\nano\meter} and \SI{0.6}{\ampere\per\watt} at \SI{950}{\nano\meter}.

\begin{figure}[H]
	\centering
	\includestandalone[mode=buildnew]{figure/diagram/pn-junction-energy-illumination}
	\caption{Energy diagram of a reverse biased p-n junction under illumination.}\label{fig:pn_junction_illumination}
\end{figure}
\Cref{fig:pn_junction_illumination} shows a p-n junction under reverse bias where a photon excites an electron-hole pair in the depletion region.
Due to the electrostatic potential, the electrons are accelerated to the right. Analogue, the holes are accelerated to the left. The photocurrent, a diode current, \Cref{eq:diode_current}, flows across the junction.

\subsection{Lateral photodiodes}

In the previous section, we discussed the transversal photoeffect associated with the p-n junction's illumination.
In addition to the transversal photoeffect, the lateral photoeffect was first discovered by W. Schottky~\cite{Schottky30} in 1930 and later rediscovered in 1957 by J. Wallmark~\cite{Wallmark57}.
In the present section, we summarize key results from Ref.~\cite{Noorlag74}, and Ref.~\cite{Woltring75}. The focus of our summary is on the most common position-sensitive photodiode design, the tetralateral photodiode.
\begin{figure}[H]
	\centering
	\includestandalone[mode=buildnew]{figure/diagram/pn-junction-lateral-cross-section}
	\caption{Cross section of a lateral photodiode.}\label{fig:lateral_photodiode_cross_section}
\end{figure}
\Cref{fig:lateral_photodiode_cross_section} depicts the cross-section of a lateral photodiode with a p-type semiconductor as the top, an n-type semiconductor as the middle layer, and a resistive material as the bottom layer.
The photodiode's common cathode is an electric contact into the top layer.
One can ground or apply a positive voltage to the common cathode. If we apply a positive voltage, we say that the photodiode is reverse-biased.
In contrast to the transversal photodiode, the lateral photodiode has two anode contacts positioned at the opposite sites embedded into the resistive layer.
An almost linear relation between the photocurrent at each of the anode contacts and the incident light spot's center-of-mass exists.
Therefore the lateral photodiode can be used to measure the spatial coordinate of an incident light spot.

The dynamics of the lateral photodiode are guided by the Lucovsky~\cite{Lucovsky60} equation,
\begin{equation}
	\nabla^2V=\frac{\rho}{w}J_s\left(e^{eV/k_BT}-1\right)-\frac{\rho_d}{w}J_p\label{eq:lucovsky_exact},
\end{equation}
wherein $V$ is the diode voltage, $\rho$ is the resistance per unit length of the resistive layer, $J_s$ is the reverse-bias saturation current and $J_p$ is the photocurrent generated due photon induced electron-hole excitation.
\Cref{eq:lucovsky_exact} can be obtained by a combination of the current density continuity equation with Ohm's law and the Schottky equation, \Cref{eq:diode_current}.

According to Ref.~\cite{Woltring75}, operation of the lateral in fully reverse-bias has the following benefits:
\begin{enumerate}
	\item Reduced signal loss.
	\item Improved response speed and resolution.
	\item Improved linearity of the position.
	\item Reduced temperature dependence.
\end{enumerate}
In electronic engineering literature, e.g., Ref.~\cite[p.~258]{Jung05}, one is sometimes discouraged from operating a photodiode under reverse-bias for the highest sensitivity as the reverse-bias increased the diode leakage (dark) current.
We believe that as long as the dark current is significantly smaller than the typical photocurrent, the photodiode should always be operated in the reverse-bias mode as recommended by Ref.~\cite{Noorlag74,Woltring75,Wang89,Hobbs11}.

Assuming a fully reverse-biased lateral photodiode, we have $eV/k_BT\ll1$ and \Cref{eq:lucovsky_exact} simplifies to a linear second order differential (Poisson) equation,
\begin{equation}
	\nabla^2V\approx-\frac{\rho}{w}\left(J_s+J_p\right)
	\label{eq:lucovsky_reverse_bias},
\end{equation}
which can be solved using variable separation and a product Ansatz.
The solution of \Cref{eq:lucovsky_reverse_bias} depends on the imposed boundary conditions.
The Dirichlet boundary condition, $V=0$, should be used if an electrical contact terminates the semiconductor. Otherwise, the Neumann boundary condition, $\pdv{V}{n}=0$, should be assumed.
\begin{figure}[H]
	\centering
	\includestandalone[mode=buildnew]{figure/diagram/pn-junction-lateral-contacts}
	\caption{Contact configurations of a lateral photodiode.}\label{fig:lateral_photodiode_contacts}
\end{figure}
In \Cref{fig:lateral_photodiode_contacts} we present three different contact configurations.
The left contact configuration, used by Wallmark, comprises four small contacts, see Ref.~\cite{Wallmark57}.
Assuming that the contact size is small compared to the photodiode's surface, we need to use the von Neumann boundary condition.
The middle contact configuration receives great attention from Noorlag~\cite{Noorlag74}.
On two opposite sides of the photodiode, an electric contact terminates the boundary while the remaining sides are left empty.
For two-dimensional position detection, one can create two electric contacts at another layer.
Mathematically we can express this configuration as a combination of Neumann and Dirichlet boundary conditions.
For the tetralateral configuration, we need to apply the Dirichlet boundary conditions for all sides.
For a rectangular tetralateral photodiode of width $l$, the explicit boundary conditions are,
\begin{align}
	V(0, y)=V(l,y)=0, && V(x,0)=V(x,l)=0.
\end{align}
Together with the initial condition that a focused light spot hits the photodiode at $(x^*,y^*)$,
\begin{equation}
	V_p(x,y, t=0)=I_p\frac{\rho}{w}\delta(x-x^*)\delta(y-y^*)
	\label{eq:lucovsky_initial},
\end{equation}
the general solution of the Lucovsky equation, \Cref{eq:lucovsky_reverse_bias}, for the tetralateral configuration reads,
\begin{equation}
	V^*_p(x,y)=I_p\frac{\rho}{w}\sum_{n\in\mathbb{Z}}\sum_{m\in\mathbb{Z}}\frac{\sin(m\pi x/l)\sin(m\pi x^*/l)\sin(n\pi y/l)\sin(n\pi y^*/l)}{(m^2+n^2)\pi^2}
	\label{eq:lucovsky_solution}.
\end{equation}
The current that flows through the $x1$ contact is given by
\begin{equation}
	I_{x1}(x^*,y^*)=\frac{w}{\rho}\int_0^l\dd{y}\eval{\pdv{V}{x}}_{x=0},
\end{equation}
respective
\begin{equation}
	I_{x1}(x^*,y^*)=\frac{2}{\pi}I_p\sum_{n\in\mathbb{Z}}\frac{\sin[(2n-1)\pi y^*/l]}{2n-1}\frac{\sinh[\abs{2n-1}\pi(1-x^*/l)]}{\sinh(\abs{2n-1}\pi)}
	\label{eq:lucovsky_current}.
\end{equation}
The solution does not disclose the linear relationship between anode current $I_{x1}$ and the incident spot light coordinate $x^*$.
Ref.~\cite[Fig.~7]{Woltring75} renders the non-linear distortion close to the boundaries as described by \Cref{eq:lucovsky_current}.
In order to show analytically that there is an almost linear relationship between $I_{x1}$ and $x^*$, we fix $y^*=l/2$, numerically evaluate the dominant terms and Taylor expand the terms to linear order,
\begin{align}
	I_{x1}(x^*,l/2)
	&=\frac{2}{\pi}I_p\sum_{n\in\mathbb{Z}}\frac{(-1)^{n+1}}{2n-1}\frac{\sinh(\abs{2n-1}\pi(1-x^*/l))}{\sinh(\abs{2n-1}\pi)}\\
	&\approx I_p\left\{0.25-0.41731\left(\frac{x}{2l}-1\right)\right\}+\mathcal{O}\left(\left(\frac{x^*}{l}-\frac{1}{2}\right)^2\right).
\end{align}
Using the difference between two opposite contacts and normalizing for the total photocurrent, we find,
\begin{equation}
	\frac{I_{x2}(x^*,l/2)-I_{x1}(x^*,l/2)}{I_{x1}(x^*,l/2)+I_{x2}(x^*,l/2)}
	\propto\frac{x}{l},
\end{equation}
which indeed is linear in $x$.

According to Ref.~\cite[p.~41]{Noorlag74}, the tetralateral photodiode has benefits in manufacturing, although its linearity is below the dual lateral photodiode, still better than the Wallmark type, see Ref.~\cite{Woltring75}.
Recent contact configurations have improved upon the tetralateral design to improve the linearity.
For example, the commercially available pin-cushion tetralateral photodiode, discussed in Ref.~\cite{Doke87,Wang89}, shows good linearity over a large area.
The center-of-mass of an incident light spot at $(x^*,y^*)$ can be recovered from the anode currents via,
\begin{align}
	\frac{\left(I_{x2}+I_{y1}\right)-\left(I_{x1}+I_{y2}\right)}{I_{x1}+I_{x2}+I_{y1}+I_{y2}}=\frac{2x^*}{l} &&
	\frac{\left(I_{x2}+I_{y2}\right)-\left(I_{x1}+I_{y1}\right)}{I_{x1}+I_{x2}+I_{y1}+I_{y2}}=\frac{2y^*}{l}	.
\end{align}
The datasheet~\cite{HamamatsuS5990} of the \gls{s5990}, a pin-cushion tetralateral photodiode, discloses the position detectability of a light spot of size \SI{0.2}{\milli\meter} over a scan interval of \SI{0.4}{\milli\meter} which does not show any non-linear distortion.

\subsection{Equivalent circuit}

In the previous section, we described the physics behind the lateral photodiode.
In the current section, we want to ignore microscopic details and concentrate on real tetralateral photodiodes' electrical properties.

In \Cref{fig:tetralateral_photodiode_equivalent} the equivalent circuit of a tetralateral photodiode is depicted.
The tetralateral photodiode has four anode connections X1, X2, Y1, and Y2 which are connected to the junction via position-dependent resistances $R_1$, $R_2$, $R_3$, and $R_4$.
The actual junction is equivalent to two current sources $I_p,I_d$, a resistor $R_d$, and a capacitor $C_d$ in parallel.
The first current source $I_p$ represents the generated photocurrent with typical values between \si{\micro\ampere} and \si{\milli\ampere}.
The second current source $I_d$ represents the generated leakage or dark current with typical values between \si{\pico\ampere} and \si{\nano\ampere}.
\begin{figure}[H]
	\centering
	\includestandalone[mode=buildnew]{figure/circuit/photodiode-tetralateral}
	\caption{Equivalent circuit of a tetralateral photodiode.}\label{fig:tetralateral_photodiode_equivalent}
\end{figure}
The resistance $R_d$ is referred to as interelectrode resistance and is about \SI{10}{\kilo\ohm}.
The capacitance $C_d$ is also referred to as terminal capacitance and is about \SI{150}{\pico\farad}.
Resistance $R_d$ and capacitance $C_d$ form an RC pole which frequency is given by,
\begin{equation}
	f_d=\frac{1}{2\pi R_dC_d}.
\end{equation}
For $R_d=\SI{10}{\kilo\ohm}$ and $C_d=\SI{100}{\pico\farad}$ the cut-off frequency $f_d$ of the pole is about \SI{1}{\mega\hertz}, representing the intrinsic bandwidth limit of the detector.

\subsection{Signal-to-noise ratio}

% TODO: check out Radeka1974 and Narayanan1997 for more details
% TODO: estimate non-linear errors

According to \cite{Woltring75}, the thermal (Johnson) noise of the resistive component of the lateral photodiode is the dominant noise source.
The thermal noise is given by,
\begin{equation}
	I_t=\sqrt{\frac{4k_BTB}{R}},
\end{equation}
wherein $B$ denotes the bandwidth to consider.
At room temperature $T=\SI{300}{\kelvin}$ and an interelectrode resistance of $R_d=\SI{10}{\kilo\ohm}$, we find a noise current density due to thermal noise of $i_t=\SI{2}{\pico\ampere\per\sqrt\hertz}$.
If we estimate for the complete bandwidth supported by the detector, we find a root-mean-squared noise current of $I_t=\SI{2}{\nano\ampere}$.
A more realistic bandwidth incoporating later analog components would be $B=\SI{10}{\kilo\hertz}$, yielding $I_t=\SI{200}{\pico\ampere}$.

% TODO: noise propagation from detector to anodes
In any case, we cannot say for sure how these noise sources propagate into position detection noise.
For instance, if the noise propagates in the same proportions over the anodes, any error should cancel out when calculating the position.