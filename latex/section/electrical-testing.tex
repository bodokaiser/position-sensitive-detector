\section{Electrical testing}

With the electrical testing, we want to confirm that the solder connections and components work as expected.
The most basic check is that the supply voltages are correct.
More elaborated checks concern the operational amplifiers but are out of the scope of this document.

\subsection{Arithmetic board I}

Leave the arithmetic board unpowered and measure the resistance between the supply lines through the test points TP1 (+12 V), TP2 (GND), and TP3 (-12 V).
There should be very high resistance between any of these test points.
Otherwise, there is a short.

If no short is detected, you can configure the power supply to provide a dual voltage of +- 15 V.
Alternatively, use two single voltage sources set to 15 V and connect the negative port of one of the voltage sources with the other one's positive voltage.
Confirm that the output voltages are correct with the multimeter. If possible, limit the current of the voltage sources.
You can set the current limit as low as possible and increase from there until the voltage becomes stable.
You might need to adjust the limit when connecting the voltages with the arithmetic board.

Connect the power supply with the arithmetic board but double-check the polarity.
The voltage between TP2 (GND) and TP1 (+12 V) should read around + 12 V while the voltage between TP2 (GND) and TP3 (- 12 V) should read about - 12 V.

Confirm that the supply voltages reach the op-amps.
Check the temperature of the op-amps.
If the op-amps are very hot, there is a short.
You can additionally check if the op-amps work by grounding the inverting input.
Because the op-amp's non-inverting input is connected with the ground, the op-amp output should be a virtual ground.

\subsection{Detector board}

Have the detector board unplugged and measure the resistance between GND (you can use the center pin of the headers), TP5 (+ 5V), and TP6 (-5 V). If the resistance is high, proceed. Otherwise, find the short.

Measure the voltage between the photodiode and GND's outer pins while illuminating the sensitive area of the photodiode with a laser pointer.
You should detect a small voltage.

If the preceding checks are successful, mount the detector board on the arithmetic board, and measure the detector board's supply voltages.

Finally, confirm that the voltage reference (U3) outputs 10 V and that the op-amps have the correct supply voltage.

\subsection{Arithmetic board II}

Leave the detector board mounted on the arithmetic board.
Measure the voltage at the outputs of the operational amplifiers of the arithmetic board.
For the summing amplifier (U5), we expect an increase in voltage when illuminating the photodiode with the laser pointer.
For the difference amplifiers (U3, U4), we expect the voltage to change when moving the laser pointer's focal spot on the photodiode.

