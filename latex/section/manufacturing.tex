\section{Manufacturing}

The manufacturing of the position-sensitive device involves the necessary steps to assemble the detector and arithmetic PCB.

It's best first to do the arithmetic board and if it works, proceed with the detector. Of course, if you only need one of the boards, perform the following steps only once.

\subsection{Component sourcing}

Before we can assemble the PCBs, we need to gather the required components and order missing pieces.

The list of the required components is usually known as the bill of materials (BOM). You can generate a current BOM directly from the Kicad files by using the InteractiveHTML plugin. The plugin generates an HTML file that lists the components and their respective placement on the PCB. Furthermore, you can check if the parts are stocked and placed. Start by checking the local inventory for the components listed by the generated HTML BOM.

If certain elements are missing, you can check the BOM.xlsx file inside the project repository. It contains part numbers and product links from major electronic distributors. If you order, double-check the part of the BOM.xlsx with the generated HTML BOM, there might be errors, e.g., wrong casing sizes in the BOM.xlsx as the BOM.xslx is older.

Finally, it would be best to have the boxes for all the required components in front of you and marking them as sourced in the generated HTML BOM.

\subsection{Component placement}

After you sourced the components, you can place them using the solder paste on the PCB.

Put on rubber gloves as the solder paste is potentially hazardous. Take out the solder paste from the fridge and let it reach room temperature. It is possible to use the cold solder paste directly, but it makes the handling more difficult as the paste is less fluid.

Place the solder paste onto the contacts of the PCB. Generally speaking, it is better to have too much solder paste then too less. If there is a small overlap between the contacts, this is fine as the solder will shrink when transitioning to a liquid phase in the oven.

Now, place the SMD components onto the solder with a tweezer. For me, it works best to start with the larger components and then progress to the smaller ones. You can check the components as placed in the generated HTML BOM.

\subsection{Reflow soldering}

Start the oven and adjust the frame on which you place the board with the screw to the appropriate size. Use an empty PCB for this.
After that, select the profile SMD180 and follow the instructions of the oven.

\subsection{Quality control}

Inspect the board for misplaced components and unmelted solder. You can use the solder heat gun to melt the remaining solder paste or remove misplaced components. However, you might blow away small capacitors.
You can fix other defects by hand soldering.

\subsection{Hand soldering}

Some components, for example, the SMA mounts or the pin headers, have to be soldered by hand. Before soldering these components by hand, jump to the electrical testing section. Only, if your board passes the electrical tests, proceed with hand soldering the missing parts.

