\section{Schematic}

The present section discusses the critical sections of the electrical schematics and should be consulted for reference if parameters, for example, bandwidth or gain, need to be adjusted.

\subsection{Photodiode frontend}

The photodiode frontend is responsible for converting the photocurrent to a voltage.
The most simple frontend, a resistor, has limited bandwidth.
Using a transimpedance amplifier, we can decouple the output voltage from the voltage swing across the diode.
The bandwidth is now limited by the amplifier gain.

\begin{figure}[H]
	\centering
	\includestandalone[mode=buildnew]{figure/circuit/amplifier-transimpedance-input-capacitance}
 	\caption{Transimpedance amplifier for photodiode}\label{fig:transimpedance_amplifier}
\end{figure}

\Cref{fig:transimpedance_amplifier} shows a transimpedance amplifier frontend circuit representing the photodiode by a current source with capacitance.
The current source relates to the power of the incident light by
\begin{equation}
I_\text{in} = S(\lambda) P,
\end{equation}
wherein $S(\lambda)$ is the spectral responsivity, also known as photosensitivity, and $P$ is the power of the light beam.
The photodiode's datasheet specifies the capacitance $C_\text{in}$.

The operational amplifier's output voltage in the transimpedance configuration is equal to
\begin{equation}
	V_\text{out} = - R_f I_\text{in}.
\end{equation}

A feedback capacitance
\begin{equation}
	C_f\geq\sqrt{\frac{C_\text{in}}{2\pi R_f f_u}},
	\label{eq:feedback_capacitance}
\end{equation}
wherein $f_u$ is the \gls{gbp} of the operational amplifier,
 is sufficient for stability.
 In general, the input capacitance of the operational amplifier adds to the input capacitance.
 In practice, solder joints and \gls{pcb} traces may add to the capacitance such that a slightly higher value than given by \eqref{eq:feedback_capacitance} is recommended.
 However, if the feedback capacitance is too high, we remove high-frequency components of our signal, which might be of interest.

\subsection{Voltage regulator}

We use two voltage regulators to maintain a constant voltage of \SI{\pm12}{\volt} from the external \SI{\pm15}{\volt} power supply. The \SI{\pm12}{\volt} voltage powers most components except the transimpedance amplifiers of the detector.
Using a dedicated pair of voltage regulators to power the transimpedance amplifiers decreases the load on the primary voltage regulator.

\Cref{fig:voltage_regulator} shows the circuitry of the primary voltage regulators that output \SI{\pm12}{\volt}.
The positive voltage regulator U1 of type LM317 and the negative voltage regulator U2 of type LM337 have an internal reference voltage of \SI{1.25}{\volt}.

\begin{figure}[H]
	\centering
	\includestandalone[mode=buildnew]{figure/circuit/voltage-regulator}
	\caption{Dual supply voltage regulator}\label{fig:voltage_regulator}
\end{figure}

The voltage divider comprising R1 and R2 fixes the output voltage according to
\begin{equation}
	V_\text{out}=\SI{1.25}{\volt}\left(1+\frac{R_2}{R_1}\right).
\end{equation}
With $R1 = \SI{240}{\ohm}$ and $R2 = \SI{2.2}{\kilo\ohm}$, the output voltage is about \SI{12.7}{\volt}, which leaves enough space for potential voltage drops in, for example, the operational amplifiers.

Diodes D1 and D3 ensure that the capacitors C3 and C5 can discharge over the external power supply.

The application notes of one of the voltage regulator's datasheets describe the presented design.
That is also the reason for not using fixed voltage regulators in this case.

\subsection{Voltage reference}

\begin{figure}[H]
	\centering
	\includestandalone[mode=buildnew]{figure/circuit/voltage-reference}
	\caption{Voltage reference to reverse bias the photodiode}\label{fig:voltage_reference}
\end{figure}

\subsection{Analog arithmetic}

\begin{figure}[H]
	\centering
	\includestandalone[mode=buildnew]{figure/circuit/amplifier-arithmetic}
	\caption{Summing and difference amplifier}\label{fig:amplifier_arithmetic}
\end{figure}
