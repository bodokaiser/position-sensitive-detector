\section{Schematic}

The present section discusses the critical sections of the electrical schematics and should be consulted for reference if parameters, for example, bandwidth or gain, need to be adjusted.

\subsection{Photodiode frontend}

The photodiode frontend is responsible for converting the photocurrent to a voltage.
The most simple frontend, a resistor, has limited bandwidth.
Using a transimpedance amplifier, we can decouple the output voltage from the voltage swing across the diode.
The bandwidth is now limited by the amplifier gain.

\begin{figure}[H]
	\centering
	\includestandalone[mode=buildnew]{figure/circuit/amplifier-transimpedance-input-capacitance}
 	\caption{Transimpedance amplifier for photodiode}\label{fig:transimpedance_amplifier}
\end{figure}

\Cref{fig:transimpedance_amplifier} shows a transimpedance amplifier frontend circuit representing the photodiode by a current source with capacitance.
The current source relates to the power of the incident light by
\begin{equation}
I_\text{in} = S(\lambda) P,
\end{equation}
wherein $S(\lambda)$ is the spectral responsivity, also known as photosensitivity, and $P$ is the power of the light beam.

A class 2 laser can be operated without safety googles.
For a Class 2 laser, the beam power is limited to $P\leq\SI{1}{\milli\watt}$. With $S=\SI{0.6}{\ampere\per\watt}$, the photocurrent is limited to $I_\text{in}\leq\SI{0.6}{\milli\ampere}$.

The operational amplifier's output voltage in the transimpedance configuration is equal to
\begin{equation}
	V_\text{out} = - R_f I_\text{in}.
	\label{eq:transimpedance_amplifier_output_voltage}
\end{equation}
The operational amplifier's output voltage is determined by it's supply voltages.
For a rail-to-rail operational amplifier it is equal to the supply voltage.
For example, if the supply voltage is \SI{\pm5}{\volt}, \eqref{eq:transimpedance_amplifier_output_voltage} gives us a value for the feedback resistance,
\begin{equation}
	R_f\geq\abs{\frac{V_\text{out}}{I_\text{in}}}=\abs{\frac{V_\text{supply}}{I_\text{in}}}
	\label{eq:feedback_resistance}.
\end{equation}
Using the previous estimated photocurrent of a class 2 laser, $I_\text{in}=\SI{0.6}{\milli\ampere}$, $V_\text{supply}=\SI{\pm5}{\volt}$, and \eqref{eq:feedback_resistance}, we find that $R_f\approx\SI{9}{\kilo\ohm}$ would make our transimpedance amplifier saturate at the maximal expected light power.

A feedback capacitance
\begin{equation}
	C_f\geq\sqrt{\frac{C_\text{in}}{2\pi R_f f_u}},
	\label{eq:feedback_capacitance}
\end{equation}
wherein $f_u$ is the \gls{gbp} of the operational amplifier,
 is sufficient for stability.
 In general, the input capacitance of the operational amplifier adds to the input capacitance.
 In practice, solder joints and \gls{pcb} traces may add to the capacitance such that a slightly higher value than given by \eqref{eq:feedback_capacitance} is recommended.
 However, if the feedback capacitance is too high, we remove high-frequency components of our signal, which might be of interest.
 
The photodiode's datasheet specifies the capacitance $C_\text{in}$ to be about \SI{90}{\pico\farad} if reverse-biased at \SI{10}{\volt}.
The differential- and common-mode input capacitance of the OPA847 operational amplifier add up to a total input capacitance of \SI{98}{\pico\farad}.
Using $R_f=\SI{10}{\kilo\ohm}$ and \gls{gbp} of $f_u=\SI{10}{\mega\hertz}$, a feedback capacitance of at least $C_f=\SI{12.5}{\pico\farad}$ should be choosen.

\subsection{Voltage regulator}

We use two voltage regulators to maintain a constant voltage of \SI{\pm12}{\volt} from the external \SI{\pm15}{\volt} power supply. The \SI{\pm12}{\volt} voltage powers most components except the transimpedance amplifiers of the detector.
Using a dedicated pair of voltage regulators to power the transimpedance amplifiers decreases the load on the primary voltage regulator.

\Cref{fig:voltage_regulator} shows the circuitry of the primary voltage regulators that output \SI{\pm12}{\volt}.
The positive voltage regulator U1 of type LM317 and the negative voltage regulator U2 of type LM337 have an internal reference voltage of \SI{1.25}{\volt}.

\begin{figure}[H]
	\centering
	\includestandalone[mode=buildnew]{figure/circuit/voltage-regulator}
	\caption{Dual supply voltage regulator}\label{fig:voltage_regulator}
\end{figure}

The voltage divider comprising R1 and R2 fixes the output voltage according to
\begin{equation}
	V_\text{out}=\SI{1.25}{\volt}\left(1+\frac{R_2}{R_1}\right).
\end{equation}
With $R1 = \SI{240}{\ohm}$ and $R2 = \SI{2.2}{\kilo\ohm}$, the output voltage is about \SI{12.7}{\volt}, which leaves enough space for potential voltage drops in, for example, the operational amplifiers.

Diodes D1 and D3 ensure that the capacitors C3 and C5 can discharge over the external power supply.

The application notes of one of the voltage regulator's datasheets describe the presented design.
That is also the reason for not using fixed voltage regulators in this case.

\subsection{Voltage reference}

Reverse biasing the photodiode decreases it's capacitance and thereby increases the bandwidth.
The reverse bias voltage needs to be well controlled not to cause any side-effects.
Therefore, we utilize a voltage reference to supply a highly-regulated reverse voltage for the photodiode.

\begin{figure}[H]
	\centering
	\includestandalone[mode=buildnew]{figure/circuit/voltage-reference}
	\caption{Voltage reference to reverse bias the photodiode}\label{fig:voltage_reference}
\end{figure}

\Cref{fig:voltage_reference} shows the schematic of the reverse-bias voltage reference.
The capacitors C1 and C3 intercept voltage fluctuations. 
Resistor R1 forms a low-pass with capacitor C2 that suppresses frequencies above 160 Hz.

\subsection{Analog arithmetic}

Finally, we need to add and subtract the four different voltage signals of the position-sensitive photodiode to obtain quantities proportional to the incident light beam's center-of-mass.
We achieve immutability of the input signals for addition and subtraction with the summing amplifier.
We extend the summing amplifier to allow negative signs, i.e., subtraction.

\begin{figure}[H]
	\centering
	\includestandalone[mode=buildnew]{figure/circuit/amplifier-arithmetic}
	\caption{Summing and difference amplifier}\label{fig:amplifier_arithmetic}
\end{figure}

\Cref{fig:amplifier_arithmetic} shows such an arithmetic amplifier.
According to \cite[p. 5]{Tietze15} the output voltage is given by
\begin{equation}
	V_\text{out}=-\frac{R_1}{R_f}V_1-\frac{R_2}{R_f}V_2+\frac{R_3}{R_f}V_3+\frac{R_4}{R_f}V_4.
	\label{eq:arithmetic_amplifier_output_voltage}
\end{equation}

It is an excellent directive to have a well-controlled the bandwidth.
On this account, we include capacitor $C_f$ to control the frequency response of the amplifier. 
For the resistor and capacitor configuration used in \cref{fig:amplifier_arithmetic}, the bandwidth is equal to
\begin{equation}
	\frac{1}{2\pi R_fC_f}\approx\SI{3}{\mega\hertz},
\end{equation}
magnitudes more than the bandwidth of our transimpedance amplifiers.
We place an additional capacitor with feedback capacitance at the non-inverting input of the operational amplifier to obtain a symmetric frequency response for the positive terms in \eqref{eq:arithmetic_amplifier_output_voltage}.

The value of the resistors is \SI{10}{\kilo\ohm} to limit the current flow.

